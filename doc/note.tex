



\subsection{Ulteriori specifiche}
\begin{itemize}
\item Viene usata una unica macchina dove sono allocate tutte le
  funzionalit\`a del lato server, ovvero il Proxy, i due web service,
  il server del database sono eseguiti tutti nello stesso nodo.
\end{itemize}


\begin{comment}
Gli elementi \verb'ricercans:ricercaStanza' e
\verb'prenotns:prenotazione' sono usati per realizzare le operazioni
di ricerca e prenotazione di una stanza, mentre quelli di
\verb'ricercans:ricercaStanzaResponse' e
\verb'prenotns:prenotazioneResponse' sono usati nella risposta alle
precedenti operazioni. Dato che l'encoding dei dati nei messaggi SOAP
\`e di tipo wrapped document-literal allora tali elementi sono il
possibile contenuto della parte Body dei messaggi SOAP scambiati
dall'applicazione.

Gli elementi di ricerca e di prenotazione di una stanza contengono le
informazioni sull'arrivo, la partenza e sulla stanza; un elemento
\verb'hotelns:stanza' contiene diverse informazioni che descrivono
una cambera d'albergo, tuttavia quelle necessarie per le operazioni di
ricerca e prenotazione sono il numero di adulti e il numero di bambini
ospitabili nella stanza, altre informazioni possono essere agginute
nella ricerca/prenotazione ma non sono obbligatorie (ad esempio il
codice del tipo della stanza nella ricerca, o l'eta dei bambini nella
prenotazione). Le altre informazioni realtive alla stanza sono
utilizzate nella risposta (ad esempio la descrizione della
stanza). L'elemento di prenotazione inoltre contiene informazioni
sull'identificatore della stanza che si vuole prenotare, e sulla
persona che effettua l'ordine e la carta di credito con cui effettuare
il pagamento. L'identificatore delle stanza deve essere uno di quelli
reperiti da precedenti risposte dell'hotel alle ricerce di stanza. Si
osserva che l'identificatore di una stanza, \verb'hotelns:stanzaId' in
'prenotazione' e 'prenotazioneResponse' \`e usato dall'hotel per
identificare univocamente una stanza con certi servizi ed un certo
costo, mentre il codice del tipo della stanza, 'codiceTipoStanza' in
identifica una macro categoria a cui la stanza appartiene.
\end{comment}


%%%%%%%%%%%%%%%%%%%%%%%%%%%%%%%%%%%%%%%%%%%%%%%%%%%%%%%%%%%%%%%%%%%%%%%%%%%%%


%%%%%%%%%%%%%%%%%%%%%%%%%%%%%%%%%%%%%%%%%%%%%%%%%%%%%%%%%%%%%%%%%%%%%%%%%%%%%



\begin{comment}
ACID atomicita consistenza isolazione durabilita

dirty read
- lettura di una riga che ha subito un aggiornamento da una
  transazione che nel futuro termina con rollback
> non si verifica mai, una operazione di una transazione non ha mai
  visione delle modifiche non committate

unrepeatable read
- due letture consecutive della stessa riga ritornano risultati
  diversi 

lost update
- una riga letta e successivamente scritta. tra tali due operazioni
  un'altra transizione aggiorna la riga stessa e committa, tale
  aggiornamento viene perso.

phantom row 
- due letture consecutive di un insieme di righe ritorna due insiemi
  diversi di risultati.

serializzable isolation level
- selezione delle righe: per tutta la transazione si lavora su una
  fotografia del database scattata all'avvio della transazione. 
- aggiornamento, inserzione, cancellazione, selezione per
  aggiornamento:  se una di queste operazioni seleziona una riga che
  \`e stata aggiornata/cancellata in precedenza (ma dopo l'inizio
  della nostra transazione) da un'altra transazione allora la nostra
  transazione si blocca in attesa della terminazione dell'altra
  transazione. Se questa termina con rollback la nostra transazione
  procede normalmente, altrimenti il dbms genera un errore dovuto al
  conflitto di serializzazione
\end{comment}

%%%%%%%%%%%%%%%%%%%%%%%%%%%%%%%%%%%%%%%%%%%%%%%%%%%%%%%%%%%%%%%%%%%%%%%%%%%%%


\subsection{Creazione dei certificati X.509 e dei java key store}
\begin{verbatim}
$ cd <baseDir>/sources/resources/keystore/
$ keytool -genkey -alias hotel -keypass hotelpassw -keystore
hotel-keystore.jks -storepass storepass -dname "cn=Hotel" -keyalg RSA
$ keytool -list -keystore hotel-keystore.jks -storepass storepass  -v

Tipo keystore: jks
Provider keystore: SUN

Il keystore contiene 1 entry

Nome alias: hotel
Data di creazione: 5-gen-2012
Tipo entry: keyEntry
Lunghezza catena certificati: 1
Certificato[1]:
Proprietario: CN=Hotel
Organismo di emissione: CN=Hotel
Numero di serie: 4f056ebd
Valido da Thu Jan 05 10:34:53 CET 2012 a Wed Apr 04 11:34:53 CEST 2012
Impronte digitali certificato:
 MD5: AA:0F:F6:9F:B8:6B:BA:18:EC:E9:02:EB:60:76:CA:7F
 SHA1: 15:B5:FF:BB:A1:23:55:B2:60:82:7E:0A:39:29:5F:D2:47:B3:E4:8A


*******************************************
*******************************************

$ keytool -export -keystore hotel-keystore.jks -storepass storepass
-alias hotel -file hotel.rsa
Il certificato ? memorizzato nel file <hotel.rsa>
$ keytool -import -file hotel.rsa -keystore alice.truststore.jks
-storepass storepass -alias hotel
Proprietario: CN=Hotel
Organismo di emissione: CN=Hotel
Numero di serie: 4f056ebd
Valido da Thu Jan 05 10:34:53 CET 2012 a Wed Apr 04 11:34:53 CEST 2012
Impronte digitali certificato:
 MD5: AA:0F:F6:9F:B8:6B:BA:18:EC:E9:02:EB:60:76:CA:7F
 SHA1: 15:B5:FF:BB:A1:23:55:B2:60:82:7E:0A:39:29:5F:D2:47:B3:E4:8A
Considerare attendibile questo certificato? [no]:  si
Il certificato ? stato aggiunto al keystore
\end{verbatim}

%%%%%%%%%%%%%%%%%%%%%%%%%%%%%%%%%%%%%%%%%%%%%%%%%%%%%%%%%%%%%%%%%%%%%%%%%%%%%

\section{Esecuzione}
\subsection{Client}
Il Client usa il framework \emph{Spring}, controllato dalla
configurazione in
\verb'hotelBaseDir/sources/'\-\verb'java/com/hotel/client/beans.xml'; in
tale file sono cablati gli indirizzi su cui sono pubblicati i due
servizi web. Per il testing dell'applicazione tali indirizzi sono
caratterizzati da host = \verb'localhost' e porta = \verb'2020'. Prima
di eseguire i processi client, localmente, viene mandato in esecuzione
un programma che faccia da proxy (ad esempio TcpMonitor) con i
parametri porta locale = \verb'2020', host remoto =
\verb'www.laiserver.com', porta remota = \verb'8080'; dove l'host e la
porta remota fanno parte degli indirizzi in cui i servizi sono
effettivamente pubblicati.

\subsection{Uso di Ant}
Per l'esecuzione con \emph{Ant} deveno essere impostate le seguenti
properties in \verb'build.xml': 
\begin{itemize}
\item cxf.classpath, 
\item commons-cli.jar.dir.
\end{itemize}
Mostra l'aiuto:
\begin{verbatim}
$ ant -buildfile <hotelBaseDir>/build.xml client -Darg1="--help"
\end{verbatim}
Esegue la ricerca:
\begin{verbatim}
$ ant -buildfile <hotelBaseDir>/build.xml client -Darg1="-o ricerca -b
2011-12-30 -e 2012-01-02 -a 2 -c 2"
\end{verbatim}
ovvero:
\begin{verbatim}
$ ant -buildfile<hotelBaseDir>/build.xml client -Darg1="
> --operation ricerca 
> --arrivo 2011-12-39 
> --partenza 2012-01-02 
> --numAdulti 2 
> --numBambini 2"
\end{verbatim}
Esegue la prenotazione
\begin{verbatim}
$ ant -buildfile <hotelBaseDir>/build.xml client -Darg1="-o prenotazione
-s 1A0 -b 2011-12-30 -e 2012-01-02 -a 2 -c 2 -p
ASDFGHJ123456789:Federico:Mariti:federico.mariti@gmail.com:1234 -C
visa:123:456:2014-01"
\end{verbatim}
ovvero:
\begin{verbatim}
$ ant -buildfile build.xml client -Darg1="
> --operation prenotazione
> --stanzaId 1A0
> --arrivo 2011-12-39 
> --partenza 2012-01-02 
> --numAdulti 2 
> --numBambini 2
> --persona
ASDFGHJ123456789:Federico:Mariti:federico.mariti@gmail.com:1234
> --cartaCredito visa:123:456:2014-01"
\end{verbatim}

\subsection{Invocazione da shell}
\begin{verbatim}
$ CP_MY=build/classes/
$ CP_CXF=/usr/share/java/apache-cxf-2.2.6/lib/
$ CP_CLI=/usr/share/java/apache-commons-cli-1.2/commons-cli-1.2.jar 
$ export CLASSPATH=$CP_MY:$CP_CLI:`allJars.sh $CP_CXF`
$ java com.hotel.client.Client --help
\end{verbatim}
Dove \verb'allJars.sh' \`e
\begin{lstlisting}[language=bash]
#!/bin/bash
RESULT=
for j in ``$@'' ; do
    for i in `ls ``$j''` ; do
        if [ ``${i##*.}'' = ``jar'' ] ; then 
            RESULT=''${RESULT}:${j}/${i}''
        fi

    done
done
echo ``$RESULT''
\end{lstlisting}
%$


\begin{comment}
\begin{thebibliography}{9}
\bibitem{ref:ricerca.wsdl}baseDir/WebContent/ricerca.wsdl
\bibitem{ref:prenotazione.wsdl}baseDir/WebContent/prenotazione.wsdl
\bibitem{ref:Proxy.java}baseDir/sources/java/com/hotel/servlet/Proxy.java
\bibitem{ref:RicercaImpl}com.hotel.ws.ricerca.RicercaImpl
\end{thebibliography}
\end{comment}

\end{document}

